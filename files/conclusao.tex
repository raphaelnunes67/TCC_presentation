\begin{frame}{Conclusões}
\small{
\begin{itemize}
\item Através do uso do metamodelo SmartLVGrid verificou-se que a adaptação do sistema legado para um sistema dotado de recursos ocorreu de forma flexível, aproveitando boa parte da estrutura pré-existente.
\item  Constatou-se que os ACUs, de fato, apresentaram-se como servidores de DRFs, inserindo novas funcionalidades a estrutura legada. Com isso, comprova-se a convergência \textit{smart building} e, consequentemente, as melhorias associadas a eficiência energética.
\item Graças as ISFs e CSFs implementadas para validação dos testes de funcionalidades, os ACUs encontram-se preparados para interagir com a camada de interoperabilidade. Com isso, a plataforma pode expandir-se para agregar ainda mais funcionalidades no contexto de \textit{smart buildings}.
\item  Esta foi a primeira implementação física da plataforma SmartLVGrid, o que torna toda documentação apresentada neste trabalho como um marco para o desenvolvimento de novos dispositivos e aplicações da plataforma, seja no âmbito de \textit{smart buildings} ou mesmo de \textit{smart grids}.
\end{itemize}
}
\end{frame}

\begin{frame}{Trabalhos Futuros}

\begin{itemize}
\item Como trabalhos futuros, sugere-se utilizar implementações de ACUs para máquinas condicionadoras de ar dentro de ambientes prediais, visando melhorias ainda mais significativas no contexto da eficiência energética.
\end{itemize}

\begin{itemize}
\item Além disso, trabalhos que apresentem soluções com outras infraestruturas de redes de comunicação, \textit{hardware} e \textit{software}, até mesmo para outros tipos de convergência tecnológica, estará contribuindo com soluções que possibilitem recursos para alcançar o paradigma \textit{smart}.
\end{itemize}
\end{frame}
